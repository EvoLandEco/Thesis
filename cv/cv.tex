\chapter*{Curriculum Vit\ae}
\addcontentsline{toc}{chapter}{Curriculum Vit\ae}
\setheader{Curriculum Vit\ae}

\hypersetup{
  linkcolor={rug-red},
  citecolor={rug-red},
  urlcolor={rug-red}
}

% ---------------------------------------------------------
% Custom commands
% ---------------------------------------------------------
\newcommand{\name}{Tianjian Qin}
\newcommand{\headline}{Theoretical Biologist \& Infectious Disease Modeler}

% Fancy bullet for itemize
\newlist{fancyitem}{itemize}{1}
\setlist[fancyitem]{
  label=\raisebox{0.1ex}{\small\color{rug-red}\faIcon{angle-right}},
  leftmargin=2.2em,
  itemsep=2pt,
  topsep=4pt
}

% Fancy bullets for experience sections
\newlist{expitem}{itemize}{1}
\setlist[expitem]{
  label=\raisebox{0.1ex}{\small\color{rug-red}\faIcon{angle-right}},
  leftmargin=1.0em,
  itemsep=3pt,
  topsep=3pt,
  font=\normalsize
}

% Section style
\titleformat{\section}
  {\large\bfseries}
  {}{0pt}{\sectionbar\color{rug-red}}

\titlespacing*{\section}{0pt}{1.4em}{0.8em}

\titleformat{\subsection}
  {\normalsize\bfseries\color{rug-red}}
  {}{0pt}{}
\titlespacing*{\subsection}{0pt}{1.0em}{0.3em}

% A simple CV entry macro
% A slightly more spacious and styled CV entry
\newcommand{\cventry}[4]{%
  \noindent
  \textbf{#1}%
  \hfill
  {\color{rug-red}\itshape #2}\\[0.25em]%
  {\small\textit{#3}}\\[0.45em]%
  #4%
  \par\vspace{0.9em}%
}

% Compact itemize
\setlist[itemize]{leftmargin=1.5em, topsep=2pt, itemsep=2pt}

\newcommand{\sectionbar}{%
  \color{rug-red}\rule[-0.3ex]{2mm}{2.1ex}\hspace{0.7em}%
}

% ---------------------------------------------------------
% Header
% ---------------------------------------------------------
\begin{center}
  {\huge\bfseries \name}\\[0.3em]
\end{center}

\vspace{0.5em}

% ---------------------------------------------------------
% Research Profile
% ---------------------------------------------------------
\section*{Research Profile}

Theoretical biologist specializing in data-driven mathematical modeling of complex biological and epidemiological systems. My work combines stochastic processes, phylogenetic and phylodynamic methods, graph and network theory, and cutting-edge AI technologies to study diversification and infectious disease dynamics.

I am increasingly focused on infectious disease epidemiology in animal populations, leveraging:
\begin{fancyitem}
  \item \textbf{Machine learning and deep learning}
        for pattern discovery, prediction, and model comparison;
  \item \textbf{Stochastic birth--death models}
        for diversification and epidemic dynamics;
  \item \textbf{Network- and graph-based models}
        for contact and movement structures;
  \item \textbf{Bioinformatics and geographic information systems}
        to link sequence, spatial, and epidemiological data;
  \item \textbf{High-performance computing and big-data frameworks}
        for large-scale simulations and inference;
  \item \textbf{Open, reproducible software and interactive web applications}
        to support research, policy, and teaching.
\end{fancyitem}

% ---------------------------------------------------------
% Research Interests
% ---------------------------------------------------------
\section*{Research Interests}

\begin{multicols}{2}
\begin{fancyitem}
  \item Contact and livestock trade networks; digital-twins of animal movement systems
  \item Stochastic epidemic and diversification processes
  \item Statistics and visualizations on dynamic networks and phylogenetic trees/networks
  \item Bayesian and likelihood-free inference; model identifiability
  \item AI augumented infectious disease modeling
  \item Open science, reproducible pipelines, and interactive teaching tools
\end{fancyitem}
\end{multicols}

% ---------------------------------------------------------
% Education
% ---------------------------------------------------------
\section*{Education}

\cventry
  {PhD in Evolutionary Life Sciences (Theoretical Biology)}
  {2019 -- present}
  {University of Groningen, Groningen Institute for Evolutionary Life Sciences (GELIFES), Netherlands}
  {%
    \textbf{Supervisors:} \textit{Prof.\ Dr.\ Rampal Etienne, Dr.\ Luis Valente \& Dr.\ Koen van Benthem}\\[0.35em]
    Research on stochastic diversification processes and neural network--based inference from phylogenetic trees. Development of the EVE model (evolutionary relatedness--dependent diversification) and simulation/inference pipelines in R, C++, and Python.
  }

\cventry
  {MSc in Ecology (Wetland Ecology and Invasion Biology)}
  {2016 -- 2019}
  {Beijing Forestry University, School of Nature Conservation, China}
  {%
    \textbf{Supervisors:} \textit{Prof.\ Dr.\ Hongli Li, Prof. Dr.\ Feihai Yu}\\[0.35em]
    Field and experimental work on wetland plant communities and biological invasions. Combined phylogenetic diversity metrics, community composition, and trait data to investigate resistance to invasion by \emph{Alternanthera philoxeroides} and other invasive species.
  }

\cventry
  {BSc in Marine Biology}
  {2012 -- 2016}
  {Nanjing Normal University, School of Life Sciences, China}
  {%
    Broad foundation in ecology and evolution, with emphasis on marine biodiversity, fieldwork, quantitative ecology, and programming.
  }
  
% ---------------------------------------------------------
% Academic and Research Experience
% ---------------------------------------------------------
\section*{Academic and Research Experience}

\cventry
  {Post-Doctoral Researcher  -- Infectious Disease Modeling}
  {Ongoing}
  {Wageningen University \& Research, Netherlands}
  {%
   \begin{expitem}
     \item Developing network-based models and non-parametric samplers of livestock trade and contact networks to construct digital twins of animal movement systems for disease-spread simulation and policy-making.
     \item Applying spectral graph theory, stochastic processes, and other mathematical tools to reveal complex temporal network structures.
     \item Developing neural network approaches for modeling spatial point patterns of the farms of the Netherlands.
     \item Developing \texttt{Transformer}-based approaches for dynamic livestock trade network contact prediction.
     \item Developing a \texttt{Python} library for advanced dynamic network statistics and visualizations.
     \item Developing \texttt{HerdLink}, an interactive web-based application for exploring and assessing the livestock trade networks of the Netherlands.
     \item Developing \texttt{R} package \texttt{HandelR}, an automated scheduler for managing research data and running scripts to support open science and reproducible research.
     \item Exploring integration of phylogenetic/diversification tools with phylodynamic thinking for zoonotic and livestock-associated pathogens.
   \end{expitem}
  }

\cventry
  {Doctoral Researcher -- Theoretical \& Computational Biology}
  {2019 -- present}
  {University of Groningen, Netherlands}
  {%
   \begin{expitem}
     \item Traveled to 43 countries, gaining first-hand experience of diverse ecosystems and cultures.
     \item Developed stochastic birth--death models (\texttt{EVE}) to investigate how evolutionary relatedness shapes speciation and extinction at multiple phylogenetic scales.
     \item Developed novel statistics, simulation algorithms and visualization tools for time--varying stochastic branching processes.
     \item Implemented high-performance simulation pipelines in \texttt{R}, \texttt{C++} and \texttt{Python}, generating large ensembles of phylogenetic trees under complex diversification scenarios.
     \item Designed neural network inference frameworks (\texttt{EvoNN}) for parameter estimation and scenario classification from phylogenetic trees.
     \item Applied approximate Bayesian computation, likelihood-based methods, and deep learning to assess identifiability, redundancy and robustness of diversification models.
     \item Extensive experience working on supercomputers and in \texttt{Unix/Linux} shell scripting.
     \item Emphasized open science: released simulation tools as open-source software with thorough documentation and reproducible workflows.
     \item Collaborated on the \texttt{C++} based \texttt{R} package \texttt{treestats} for efficient computing of summary statistics on phylogenies.
     \item Collaborated on the \texttt{R} package \texttt{DDD} (Diversity-Dependent Diversification) for implementing reliable maximum-likelihood estimation optimizers.
     \item Collaborated on the \textit{Brassicaceae} Tree of Life project and developed \texttt{JavaScript}-based phylogenetic manipulation tool \texttt{miniape}.
   \end{expitem}
  }

\cventry
  {Researcher in Ecology \& Invasion Biology (MSc)}
  {2016 -- 2019}
  {Beijing Forestry University, China}
  {%
   \begin{expitem}
     \item Conducted extensive field surveys across 18 provinces in China to quantify wetland plant communities and invasion patterns.
     \item Developed experience with GIS tools by modeling invasive species occurrences at country scale.
     \item Established automated pipelines of leaf image analysis and phylogenetic tree reconstruction for the research group. 
     \item Analyzed how phylogenetic diversity and species richness affect community resistance to invasion along latitudinal gradient.
     \item Performed greenhouse experiments manipulating artificial plant communities to study functional and phylogenetic interactions among native species and \emph{A.~philoxeroides}.
   \end{expitem}
  }

\section*{Grants, Scholarships, and Awards}
\subsection*{Grants}
\begin{expitem}
\item \textbf{Implementatie-impuls COVID-19 programma (ZonMw)} (Mar 2026 -- Dec 2026)\\
      Co-applicant. Project: \emph{SSS-mod Paraatheidspakket bij toekomstige respiratoire uitbraken}.
  \item \textbf{Small Compute Applications (NWO)} (2025 -- 2026)\\
        Received to support high-performance computing for data-intensive modeling work.
\end{expitem}
\subsection*{Scholarships}
\begin{expitem}
  \item \textbf{CSC--RUG Joint Scholarship} (2019 -- 2023)\\
        Competitive scholarship to support PhD research in theoretical biology.
  \item \textbf{First-Class Academic Performance Scholarship (Graduate)} (2018 -- 2019)\\
        In recognition of outstanding performance within the graduate cohort.
  \item \textbf{First-Class Academic Performance Scholarship (Graduate)} (2017 -- 2018)\\
        In recognition of outstanding performance within the graduate cohort.
\end{expitem}
\subsection*{Awards}
\begin{expitem}
  \item \textbf{Denise Kirschner Best Student Paper Prize}, Journal of Theoretical Biology (2025)\\
        Inaugural best graduate student paper prize for: Qin, T., Valente, L., \& Etienne, R.S. (2025).
  \item \textbf{Graduate Academic Innovation Award}, Beijing Forestry University (2018)\\
        Awarded for innovative graduate research.
\end{expitem}

% ---------------------------------------------------------
% Professional Skills
% ---------------------------------------------------------
\section*{Professional Skills}

\subsection*{Quantitative and Modeling Skills}

\begin{expitem}
  \item Stochastic birth--death and branching processes.
  \item Network and graph-based modeling.
  \item Machine learning and deep learning.
  \item Approximate Bayesian computation, likelihood-based inference, Markov chain Monte Carlo.
  \item Phylogenetic and phylodynamic analysis; tree and network statistics.
  \item Geospatial analysis and statistics; spatial point pattern modeling.
\end{expitem}

\subsection*{Programming and Technical Skills}

\begin{expitem}
  \item \textbf{R}: package development, simulation, data analysis, visualization.
  \item \textbf{C/C++}: high-performance simulation engines, R/Python integration.
  \item \textbf{Python}: ML pipelines, automation scripts, graph/network analysis.
  \item \textbf{Web}: HTML, CSS, JavaScript, D3.js for interactive teaching and outreach tools.
  \item \textbf{HPC \& Linux}: Environment configuration, bash scripting, job scheduling, pipeline automation.
  \item \textbf{Git \& CI/CD}: Git/GitLab workflows, GitHub Actions for automated testing and pipelines.
  \item \textbf{Server}: deploying and maintaining websites and online game servers on cloud-based Linux servers.
  \item \LaTeX\ and visualization tools (TikZ, PGFPlots) for scientific writing and figures.
\end{expitem}

% ---------------------------------------------------------
% International and Interdisciplinary Experience
% ---------------------------------------------------------
\section*{International and Interdisciplinary Experience}

\begin{expitem}
  \item Fieldwork across multiple provinces in China during MSc training, including wetland biodiversity and invasion surveys.
  \item Research and collaboration experience at the interface of ecology, evolution, statistics, and computer science.
  \item Comfortable working in multicultural academic environments and interdisciplinary teams.
\end{expitem}

% ---------------------------------------------------------
% Languages
% ---------------------------------------------------------
\section*{Languages}

\begin{expitem}
  \item \textbf{Chinese}: native
  \item \textbf{English}: proficient (IELTS Certificate C1, 2019).
  \item \textbf{Dutch}: intermediate (DUO \emph{inburgeringsdiploma} A2, 2025).
  \item \textbf{German}: reading and basic communication (Summer Camp A2.2, 2013).
\end{expitem}

% ---------------------------------------------------------
% Personal Interests
% ---------------------------------------------------------
\section*{Personal Interests}

\begin{expitem}
  \item Photography and digital storytelling.
  \item Travel and cultural exploration.
  \item Continuous learning at the interface of biology, mathematics, and computer science.
\end{expitem}

