\begin{figure}[ht]
  \centering
  \begin{tikzpicture}[
    node distance=1.4cm,
    tip/.style={circle,inner sep=1pt},
    >=Stealth
  ]

  % Background highlight for conflicting area
  \fill[pastelpink] (0.4,0.0) -- (3.0,0.0) -- (3.0,2.0) -- (0.4,2.0) -- cycle;

  % Nodes
  \node[tip] (A) at (0,1.5) {};
  \node[tip] (B) at (0,0.5) {};
  \node[tip] (C) at (3.4,1.5) {};
  \node[tip] (D) at (3.4,0.5) {};

  % Labels
  \node[accentred,anchor=east] at (A.west) {taxon A};
  \node[accentred,anchor=east] at (B.west) {taxon B};
  \node[accentred,anchor=west] at (C.east) {taxon C};
  \node[accentred,anchor=west] at (D.east) {taxon D};

  % Internal junctions for two alternative splits
  \coordinate (L1) at (0.8,1.0);
  \coordinate (R1) at (2.6,1.0);
  \coordinate (L2) at (0.8,1.8);
  \coordinate (R2) at (2.6,0.2);

  % Split 1: (A,B)|(C,D)
  \draw[treegray,thick]
    (A) -- (L1)
    (B) -- (L1)
    (C) -- (R1)
    (D) -- (R1)
    (L1) -- (R1);

  % Split 2: (A,D)|(B,C) forming the box
  \draw[accentorange,thick,dashed]
    (A) -- (L2) -- (R2) -- (D) -- cycle;

  % Annotation
  \node[rug-red,anchor=south west,font=\sffamily\footnotesize] at (0.5,2.05)
    {Conflicting splits};
  \node[treegray,anchor=north,font=\sffamily\scriptsize] at (1.7,-0.2)
    {split network (data conflict as boxes)};

  \end{tikzpicture}
  \caption{Toy split network for four taxa. Two incompatible splits form a ``box'', illustrating conflicting phylogenetic signal that cannot be represented on a single bifurcating tree.}
  \label{fig::intro::split-net}
\end{figure}
