\newacronym{sr}{SR}{Species richness, a measure of the number of different species present in a particular area or ecological community. It is a simple count of species}

\newacronym{er}{ER}{Evolutionary relatedness, a measure of evolutionary relationships between species. In our study, ER can be measured as PD, ED or NND (see definitions below)}

\newacronym{pd}{PD}{Phylogenetic diversity, a measure of biodiversity that takes into account the evolutionary relationships between species. Specifically, PD is the minimum total length of all the phylogenetic branches required to span a given set of taxa on the phylogenetic tree. It provides insight into the evolutionary history represented by a group of species \citep{faith_conservation_1992}}

\newacronym{ed}{ED}{Evolutionary distinctiveness, a measure of how unique or distinct a species is in terms of its evolutionary history. ED is computed by summing the pairwise distances between each lineage and all other lineages, divided by the number of lineages minus one. A species with high ED has fewer close relatives and represents a larger amount of independent evolutionary history than a species with low ED \citep{cadotte_rarest_2010}}

\newacronym{nnd}{NND}{Nearest neighbor distance, a measure of how distinct a species is in terms of its evolutionary history with respect to the most closely related species. NND is computed by taking the branch length distance on a phylogenetic tree from one species to its nearest neighbor on the tree. While ED measures distinctiveness on a global phylogenetic level, NND measures it on a local phylogenetic level \citep{webb_exploring_2000}}

\newacronym{jone}{J One}{J One balance index, a measure of the degree of balance of a phylogeny. It is defined for trees with any degree distribution, and enables meaningful comparison of trees with different numbers of tips \citep{lemant_robust_2022}}

\newacronym{gamma}{Gamma}{Gamma statistic, a measure of internal node distribution. The value of gamma is calculated based on the distribution of node heights. A negative gamma value suggests a decrease in diversification rate over time, while a positive gamma value indicates an increase in diversification rate \citep{pybus_testing_2000}}

\newacronym{mbl}{MBL}{Mean branch length, a measure of average evolutionary change or divergence represented in a phylogenetic tree. It is calculated by summing the lengths of all the branches in a phylogenetic tree and dividing by the total number of branches \citep{webb_phylogenies_2002}}

\newacronym{mpd}{MPD}{Mean pairwise distance, a measure of the average pairwise phylogenetic distance between species in a clade. It calculates the average of all pairwise distances between species in a phylogenetic tree, providing a measure of the overall phylogenetic spread \citep{webb_phylogenies_2002}}

\newacronym{rogers}{Rogers}{Rogers balance index, a measure of "patristic distance", which represents the sum of branch lengths between two taxa in a phylogenetic tree which reflects the total amount of change along the evolutionary path between two taxa \citep{rogers_central_1996}}

\newacronym{speciation_evenness}{Speciation rate evenness}{a measure of how evenly the speciation rates are distributed on the tips of a phylogeny, using an approach originally proposed by \citet{helmus_phylogenetic_2007}, which is similar in concept to Pielo's evenness index in community ecology. Its value is between 0 and 1. Values less than 1 represent increasing unevenness of speciation rates. It is also sensitive to tree topology: if the tips of a tree have the same speciation rates, the evenness value will be 1 (maximum) only when the tree is star-like}

\renewcommand*{\glstextformat}[1]{\textcolor{black}{#1}}