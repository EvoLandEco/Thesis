\chapter*{List of Publications}
\addcontentsline{toc}{chapter}{List of Publications}
\setheader{List of Publications}
\label{publications}

% Section style
\titleformat{\section}
  {\large\bfseries}
  {}{0pt}{\sectionbar\color{rug-red}}

\titlespacing*{\section}{0pt}{1.4em}{0.8em}

\titleformat{\subsection}
  {\normalsize\bfseries\color{rug-red}}
  {}{0pt}{}
\titlespacing*{\subsection}{0pt}{1.0em}{0.3em}

\section*{Peer-Reviewed Articles}

\begin{enumerate}[leftmargin=1.5em]

  \item \textbf{Qin, T.}, van Benthem, K., Valente, L.\textsuperscript{\dag}, \& Etienne, R.\textsuperscript{\dag} (2025).
        Parameter estimation from phylogenetic trees using neural networks and ensemble learning.
        \emph{Systematic Biology}.

  \item \textbf{Qin, T.}, Valente, L.\textsuperscript{\dag}, \& Etienne, R.\textsuperscript{\dag} (2025).
        Impact of evolutionary relatedness on species diversification and tree shape.
        \emph{Journal of Theoretical Biology}.

  \item Sun, K., Liu, X.-S., \textbf{Qin, T.-J.}, Jiang, F., Cai, J.-F., Shen, Y.-L., A, S.-H., \& Li, H.-L. (2021).
        Relative abundance of invasive plants more effectively explains the response of wetland communities to different invasion degrees than phylogenetic evenness.
        \emph{Journal of Plant Ecology}.

  \item \textbf{Qin, T.}, Zhou, J., Sun, Y., Müller-Schärer, H., Luo, F., Dong, B., Li, H., \& Yu, F.-H. (2020).
        Phylogenetic diversity is a better predictor of wetland community resistance to \emph{Alternanthera philoxeroides} invasion than species richness.
        \emph{Plant Biology}.

  \item Wan, J.-Z., Wang, M.-Z., \textbf{Qin, T.}, Bu, X.-Q., Li, H.-L., \& Yu, F.-H. (2019).
        Spatial environmental heterogeneity may be the driver of functional trait variation in \emph{Hydrocotyle vulgaris} (Araliaceae), an aquatic plant invader.
        \emph{Aquatic Biology}.

  \item \textbf{Qin, T.-J.}* , Guan, Y.-T.* , Quan, H., Dong, B.-C., Luo, F.-L., Zhang, M.-X., Li, H.-L., \& Yu, F.-H. (2019).
        Growth traits of the exotic plant \emph{Hydrocotyle vulgaris} and the evenness of resident plant communities are mediated by community age, not species diversity.
        \emph{Weed Research}.

  \item \textbf{Qin, T.-J.}, Guan, Y.-T., Zhang, M.-X., Li, H.-L., \& Yu, F.-H. (2018).
        Sediment type and nitrogen deposition affect the relationship between \emph{Alternanthera philoxeroides} and experimental wetland plant communities.
        \emph{Marine and Freshwater Research}.

  \item Liu, L., Guan, Y.-T., \textbf{Qin, T.-J.}, Wang, Y.-Y., Li, H.-L., \& Zhi, Y.-B. (2018).
        Effects of water regime on the growth of the submerged macrophyte \emph{Ceratophyllum demersum} at different densities.
        \emph{Journal of Freshwater Ecology}.

\end{enumerate}

\section*{Preprints and Under Review}

\begin{itemize}
  \item None currently (all recent preprints have been published).
\end{itemize}

\section*{Manuscripts in Preparation}

\begin{itemize}
  \item \textbf{Qin, T.*}, van der Schaaf, M. T.* , Lequime, S. J. J., \& van Benthem, K.
        Fitting transmission chains using neural networks.
        Manuscript in preparation.
  \item \textbf{Qin, T.}, van Benthem, K., Valente, L.\textsuperscript{\dag}, \& Etienne, R.\textsuperscript{\dag}
        Identifying evolutionary relatedness effects on diversification from phylogenies using neural networks.
        Manuscript in preparation.
  \item \textbf{Qin, T.}, Atamer Balkan, B., Schmid, B., \& ten Bosch, Q.
        Towards better modeling and evaluation of synthetic livestock movement networks.
        Manuscript in preparation.
\end{itemize}

\noindent\textit{* indicates authors who contributed equally to this work; \textsuperscript{\dag} indicates joint senior authors.}

