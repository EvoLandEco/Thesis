\chapter{Conclusion}
\label{conclusion}

\dropcap{T}his thesis set out to understand what phylogenies of extant species can and cannot tell us about the processes that generated them. Across three chapters, I focused on two intertwined themes: (i) how ecological limits and diversity feedback can be expressed not only through species richness but also through evolutionary relatedness, and (ii) how far modern inference tools---from likelihood-based estimators to neural networks---can reliably recover diversification mechanisms and parameters from the information encoded in reconstructed trees.

Taken together, these chapters support three overarching conclusions about diversification inference from extant phylogenies under ecological limits and relatedness dependence.

\begin{fancyenumerate}
    \item! \textbf{Evolutionary relatedness offers a principled extension of ecological limits models, but its effects are scale-dependent.}
    Relatedness can modulate interactions among lineages in ways that are not captured by richness alone. Yet ``relatedness dependence'' is not a single hypothesis: lineage-specific and clade-wide effects generate qualitatively different macroevolutionary patterns, and they should be treated as distinct modeling assumptions rather than interchangeable parameterizations.

    \item! \textbf{Inference is constrained less by the choice of estimator than by the information content of trees.}
    Likelihood-based estimators and neural networks can both perform well when the process potentially leaves a clear imprint, and both fail when the imprint is weak. Tree size and effect strength set hard limits: when multiple mechanisms yield nearly indistinguishable trees, improvement on the estimators may not fully compensate for missing information.

    \item! \textbf{Model flexibility comes with an identifiability cost that manifests as conservative, mean-regressing predictions.}
    As models become more richly parameterized, the mapping from parameters and scenarios to observed tree patterns flattens over much of parameter space. This produces systematic shrinkage in regression and asymmetric confusion in classification, providing a practical warning sign: apparent predictive accuracy can hide broad non-identifiability.
\end{fancyenumerate}
